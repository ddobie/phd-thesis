\chapter{Published Notices}
\label{appendix:notices}
\section{GCN Circulars}
\begingroup
\footnotesize
\gcntitle{21625}
\begin{verbatim}
TITLE:   GCN CIRCULAR
NUMBER:  21625
SUBJECT: LIGO/Virgo G298048: ASKAP follow up
DATE:    17/08/21 00:58:33 GMT
FROM:    Tara Murphy at U of Sydney  <tara.murphy@sydney.edu.au>

D. Dobie (University of Sydney), A. Hotan (CSIRO), K. Bannister (CSIRO),
D. Kaplan (UWM), T. Murphy (University of Sydney), C. Lynch (University
of Sydney), on behalf of the ASKAP/VAST collaboration.

We are observing the LIGO localisation region (LVC GCN 21513) with the
Australian Square Kilometre Array Pathfinder (ASKAP) at a central
frequency of 1345 MHz with a bandwidth of 192 MHz. As the array is still
undergoing commissioning, we are using a nominal 12 (of 36) antennas,
although other site activities may cause that number to vary from one
observation to the next.

Our observing strategy consists of 3 pointings containing approximately
90% of the LIGO localisation region. Each pointing consists of a 5.5 x 5.5
degree square grid of 36 beams, though the exact number captured may
also vary. The first pointing is centered at

RA = 13:09:21.60
Dec = -25:00:00.00

and contains ~48% of the localisation region and the positions of 35
target galaxies (Cook et al. LVC GCN 21519) including NGC 4993, the
host of the possible optical counterpart SSS17a
(Coulter et al. LVC GCN 21529). Observations began at 2017-08-19
05:34:32 (UT) and are ongoing.

Processing and analysis of the first pointing is underway.

Thank you to CSIRO staff for supporting these observations.
\end{verbatim}
\pagebreak
\gcntitle{21639}
\begin{verbatim}
TITLE:   GCN CIRCULAR
NUMBER:  21639
SUBJECT: LIGO/Virgo G298048: ASKAP observations of SSS17a and NGC 4993 at 1.345 GHz
DATE:    17/08/22 07:23:04 GMT
FROM:    Tara Murphy at U of Sydney  <tara.murphy@sydney.edu.au>

D. Dobie (University of Sydney), A. Hotan (CSIRO), K. Bannister (CSIRO),
T. Murphy (University of Sydney), D. Kaplan (UWM), C. Lynch (University of
Sydney),
on behalf of the ASKAP/VAST collaboration.

We have observed the LIGO/Virgo localisation region (LVC GCN 21527)
with the Australian Square Kilometre Array Pathfinder (ASKAP) at a
central frequency of 1.345 GHz with a bandwidth of 192 MHz. Observations
started at 2017-08-19 05:34 UT and ended at 2017-08-19 07:58 UT.

We do not detect any emission at the position of the possible optical
counterpart SSS17a (Coulter et al. LVC GCN 21529), or its candidate
host galaxy NGC 4993. The upper limit for NGC 4993 is consistent with
a spectral index of -0.7 found by fitting the measured results from
the ATCA (Bannister et al. LVC GCN 21559) between 8.5 and 21.2 GHz.

Further analysis of our observations is ongoing.

Thank you to CSIRO staff for supporting these observations.
\end{verbatim}
\vspace{50pt}
\gcntitle{23139}
\begin{verbatim}
TITLE:   GCN CIRCULAR
NUMBER:  23139
SUBJECT: LIGO/Virgo GW170817: A steep decline in the radio light curve and
prediction for the X-rays
DATE:    18/08/13 18:38:03 GMT
FROM:    Tara Murphy at U of Sydney  <tara.murphy@sydney.edu.au>

D. Dobie (University of Sydney), K. Mooley (Caltech),
T. Murphy (University of Sydney), D. Kaplan (UWM), E. Lenc (CSIRO),
A. Corsi (TTU), D. Frail (NRAO), report on behalf of a larger collaboration

Our continued observations of GW170817 with the ATCA and the VLA up to 300
days post-merger (Mooley et al. in prep) confirm the t^(-2) decline in the
radio light curve initially reported in Mooley et al. 2018 (arXiv:1806.09693).
Such a slope rules out a cocoon-dominated outflow at late times, and is
instead the classic signature of a relativistic jet, consistent with the
VLBI result from Mooley et al. 2018. The t^(-2) decline is also expected in
the X-ray light curve, and may be confirmed by the Chandra observation carried
out on 2018 Aug 10.
\end{verbatim}
\pagebreak
\gcntitle{25445}
\begin{verbatim}
TITLE:   GCN CIRCULAR
NUMBER:  25445
SUBJECT: LIGO/Virgo S190814bv: No radio counterpart detected in ASKAP observations
DATE:    19/08/22 07:57:20 GMT
FROM:    Dougal Dobie at VAST  <ddobie94@gmail.com>

Dougal Dobie (University of Sydney/CSIRO), Adam Stewart (University of Sydney),
Ziteng Wang (University of Sydney), Tara Murphy (University of Sydney),
Emil Lenc (CSIRO), David Kaplan (UWM), Aidan Hotan (CSIRO),
Kunal Mooley (NRAO, Caltech), Gregg Hallinan (Caltech),
David McConnell (CSIRO), Julie Banfield (CSIRO), Wasim Raja (CSIRO),
Matthew Whiting (CSIRO), Vanessa Moss (CSIRO), Igor Andreoni (Caltech)
and the OzGrav, JAGWAR and GROWTH collaborations.

We report observations of the localisation region of S190814bv (LVC, GCN 25324)
with the Australian Square Kilometre Array Pathfinder (ASKAP) on 2019-08-16 at
a central frequency of 943 MHz with a bandwidth of 288 MHz.

We have observed a single 30 sq. deg. field centered on
    RA = 00:50:37.5
    Dec = -25:16:57.4
which covers approximately 85% of the sky localisation from the
LALInference skymap (GCN 25333), with a median rms of 34 uJy.

We have searched for radio emission within 5 arcseconds of the location of
the 124 optical transients reported on the Transient Name Server by the
DECam-GROWTH and DESGW teams between 2019-08-16 and 2019-08-22 as at
2019-08-22 03:00 UTC. We report coincident compact radio emission at the
location of 14 of them:

|       Name |         RA |       Dec | Int. Flux (uJy) | err. (uJy) | Notes |
| AT 2019nqa | 00:52:39.1 | -25:00:15 |             258 |         30 |       |
| AT 2019nqy | 00:56:23.2 | -24:41:11 |             393 |         29 |       |
| AT 2019nqz | 00:46:46.5 | -24:20:12 |             870 |         30 |   (a) |
| AT 2019nsr | 00:57:27.6 | -26:16:44 |             290 |         36 |   (c) |
| AT 2019nto | 00:42:03.5 | -24:48:19 |             342 |         28 |   (c) |
| AT 2019nuk | 00:54:57.9 | -26:08:03 |             233 |         28 |   (b) |
| AT 2019nul | 00:55:16.4 | -26:56:35 |             204 |         28 |   (b) |
| AT 2019nun | 00:56:48.7 | -24:54:31 |             377 |         29 | (b,c) |
| AT 2019nuo | 00:56:03.9 | -23:18:15 |             388 |         36 |   (c) |
| AT 2019nup | 00:55:04.3 | -26:46:12 |             446 |         33 |   (c) |
| AT 2019nzj | 00:52:05.3 | -26:11:03 |             759 |         29 |       |
| AT 2019nzn | 00:55:19.9 | -24:09:29 |             233 |         32 |   (c) |
| AT 2019oay | 00:45:25.2 | -25:53:43 |             348 |         31 |       |
| AT 2019ocs | 01:00:11.4 | -25:53:22 |             352 |         29 |       |

(a) reported in GCN25391
(b) reported in GCN25393
(c) source possibly extended
\end{verbatim}
\pagebreak
\begin{verbatim}
We have also performed a preliminary search for transients using TraP
(Swinbank et al. 2015), comparing this observation to the Rapid ASKAP Continuum
Survey (RACS, [1]) at a detection threshold of 0.95 mJy, corresponding to 5
times the lowest rms of the RACS image.

We find one candidate transient located at
    RA = 00:54:34.6 +/- 0.02 arcsec
    Dec = -28:02:35.3 +/- 0.01 arcsec
which we note is outside the 95% confidence region of S190814bv. We measure a 
flux density of 3.4 mJy in this observation and a local rms noise of 0.25 mJy
in the 888 MHz RACS image observed on 2019-04-26. We measured an integrated flux
density of 0.74 mJy in the RACS image using TraP. We also note that there is a
radio source coincident with this location in the image from the Very Large
Array Sky Survey (VLASS) observed on 2019-06-29 with a flux density of ~1.6 mJy
at 3 GHz.

We conducted follow-up of this source with the Australia Telescope Compact
Array (ATCA) on 2019-08-21 with two 2048 MHz bands centered on 5.5 and 9 GHz. We
measure preliminary flux densities of 2.88 +/- 0.03 mJy and 2.93 +/- 0.02 mJy
at 5.5 and 9 GHz, with respective in-band spectral indices of +0.17 and -0.37.

Combining the near-contemporaneous ATCA and ASKAP measurements we find a flat
spectral index. Based on these observations this candidate is likely to be an
unrelated AGN.

The ASKAP observation is publicly available on the CSIRO ASKAP Science
Data Archive [2] under Scheduling Block ID 9602.

Further analysis of this ASKAP observation is ongoing and
further epochs are planned.

Thank you to CSIRO staff for supporting these observations.

[1] https://www.atnf.csiro.au/content/racs
[2] https://casda.csiro.au/
\end{verbatim}
\pagebreak
\gcntitle{25472}
\begin{verbatim}
TITLE:   GCN CIRCULAR
NUMBER:  25472
SUBJECT: LIGO/Virgo S190814bv: No radio counterpart to DG19wxnjc/AT2019npv
in ASKAP observations
DATE:    19/08/25 04:26:24 GMT
FROM:    Dougal Dobie at VAST  <ddobie94@gmail.com>

Dougal Dobie (University of Sydney/CSIRO), Tara Murphy (University of Sydney),
Emil Lenc (CSIRO), David Kaplan (UWM), Aidan Hotan (CSIRO), Adam Stewart
(University of Sydney) and the OzGrav, JAGWAR and GROWTH collaborations.

We report a non-detection of the optical transient DG19wxnjc/AT2019npv
(GCN 25393) in two recent observations with the Australian Square
Kilometre Array Pathfinder (ASKAP) at a frequency of 943 MHz.

The 3-sigma upper limits at the position of this source are 75 uJy on
2019-08-16 and 96 uJy on 2019-08-23.

Both observations were a 30 sq. deg. field of the localisation region
of S190814bv (LVC, GCN 25324) centered on
    RA = 00:50:37.5
    Dec = -25:16:57.4
with a central frequency of 943 MHz with a bandwidth of 288 MHz,
as reported in GCN 25445.

Further analysis of these ASKAP observations are ongoing
and future epochs are planned.

Thank you to CSIRO staff for supporting these observations.
\end{verbatim}
\pagebreak
\gcntitle{25621}
\begin{verbatim}
TITLE:   GCN CIRCULAR
NUMBER:  25621
SUBJECT: LIGO/Virgo S190814bv: ATCA observation of ASKAP J005547-270433/AT2019osy
DATE:    19/09/03 01:20:02 GMT
FROM:    Dougal Dobie at VAST  <ddobie94@gmail.com>

Dougal Dobie (University of Sydney/CSIRO), Emil Lenc (CSIRO),
Ian Brown (UWM), Tara Murphy (University of Sydney), Adam Stewart
(University of Sydney), David Kaplan (UWM), Kunal Mooley (Caltech),
Gregg Hallinan (Caltech) and the OzGrav, JAGWAR and GROWTH collaborations.

We observed ASKAP J005547.4-270433/AT2019osy (GCN 25487) with the
Australia Telescope Compact Array (ATCA) on 2019 August 29
from 14:00-22:00 UT.

We report flux densities of:

  369+/-23 uJy at 5.0 GHz
  335+/-19 uJy at 6.0 GHz
  307+/-15 uJy at 8.5 GHz
  278+/-14 uJy at 9.5 GHz

Fitting a power-law to these values we find a spectral index of
alpha=-0.39+/-0.11, consistent with the value of -0.42 found by
Mooley et al. (GCN 25539).

Comparing to the VLA observations reported by Mooley et al. (GCN 25539),
we find that the flux density has increased by ~40% in 1.5 days across
4.5-10 GHz.

Additional epochs are required to separate intrinsic variability from
contamination from propagation effects and systematics due to differing
spatial resolution between the ATCA, VLA and ASKAP.

Thank you to CSIRO staff for supporting these observations.
\end{verbatim}
\pagebreak
\gcntitle{25691}
\begin{verbatim}
TITLE:   GCN CIRCULAR
NUMBER:  25691
SUBJECT: LIGO/Virgo S190814bv: ATCA monitoring of ASKAP J005547-270433/AT2019osy
DATE:    19/09/10 00:30:43 GMT
FROM:    Dougal Dobie at VAST  <ddobie94@gmail.com>

Dougal Dobie (University of Sydney/CSIRO), Emil Lenc (CSIRO),
Ian Brown (UWM),Tara Murphy (University of Sydney), Adam Stewart
(University of Sydney), David Kaplan (UWM), Kunal Mooley (Caltech),
Gregg Hallinan (Caltech) and the OzGrav, JAGWAR and GROWTH collaborations.

We observed ASKAP J005547.4-270433/AT2019osy (GCN 25487) with the
Australia Telescope Compact Array (ATCA) on 2019 September 6
from 12:30-19:30 UT.

We report flux densities of:
  380+/-21 uJy at 5.0 GHz
  353+/-17 uJy at 6.0 GHz
  299+/-14 uJy at 8.5 GHz
  234+/-14 uJy at 9.5 GHz

Which are consistent with the measurements obtained with the ATCA on
2019 August 29, 8 days prior (GCN 25487, GCN 25621, GCN 25539). Therefore
the previously observed variability is unlikely to be related to S190814bv.

Thank you to CSIRO staff for supporting these observations.
\end{verbatim}
\pagebreak
\gcntitle{27516}
\begin{verbatim}
TITLE:   GCN CIRCULAR
NUMBER:  27516
SUBJECT: GRB 200405B: ATCA follow-up and ASKAP limits on pre-burst radio emission
DATE:    20/04/10 01:56:01 GMT
FROM:    Dougal Dobie at VAST  <ddobie94@gmail.com>

Dougal Dobie (USYD/CSIRO), Tara Murphy (USYD), James Leung (USYD/CSIRO),
Adam Stewart (USYD), Joshua Pritchard (USYD), David Kaplan (UWM)

GRB 200405B (GCN 27497) occurred in a field that has been observed 5 times
as part of the ASKAP Variables And Slow Transients (VAST; Murphy et al. 2013)
pilot survey between 2019-08-27 and 2020-01-25. We have searched for radio
emission from the BAT ground-calculated position and the 4 uncatalogued
X-ray sources detected by Swift (GCN 27500) and find no radio counterparts
to a detection limit of ~1.5 mJy at 888 MHz.

We also performed follow-up observations of all 5 positions with the ATCA
between 2020-04-09 05:00-10:30 UTC with 2x2 GHz bands centered on
5.5 and 9 GHz. No radio emission was detected coincident with any of the
sources, we list 3-sigma upper limits below

Source      5.5 GHz (uJy)    9 GHz (uJy)
BAT pos                81             54
Source 1               90            120
Source 2               84             96
Source 3               87             90
Source 4               90            114

We do detect a radio source at coordinates of 04:10:26.8, -51:31:55
(offset 7.2 arcsec from Source 4), coincident with WISEA J041026.82-513155.2,
with a flux density of ~4 mJy at 5.5 GHz and ~7 mJy at 9 GHz. This source is
also detected in the VAST pilot survey with a flux density of ~10 mJy. We do
not consider this a candidate afterglow for the GRB.

Further observations with ATCA and as part of the VAST program are planned.

Thank you to CSIRO staff for supporting these observations during these
especially difficult times.
\end{verbatim}
\endgroup
\pagebreak
\section{Astronomer's Telegrams}

\begingroup
\footnotesize
\ateltitle{11795}{ATCA Observations of AT2018cow}
\begin{center}
\textit{Dougal Dobie (University of Sydney), Vikram Ravi (Caltech), Anna Ho (Caltech),\\
Mansi Kasliwal (Caltech), Tara Murphy (University of Sydney)\\
on 29 Jun 2018; 15:52 UT\\
Credential Certification: Dougal Dobie (ddob1600@uni.sydney.edu.au)}
\end{center}

\begin{verbatim}
We have observed the position of AT2018cow (ATel #11727) with the
Australia Telescope Compact Array (ATCA) on 2018-06-26 from 09:00-14:00 UT
at central frequencies of 5.5 GHz, 9 GHz with a bandwidth of 2 GHz in each
band, and 34 GHz with a bandwidth of 4 GHz. Due to maintenance, only four
of six antennas were available, and only three antennas were used at 34 GHz.

We detect a point source coincident with the location of AT2018cow (ATel #11727)
and report preliminary flux densities of:

<0.18 mJy at 5.5 GHz (3-sigma upper limit);
~0.46 mJy at 9 GHz;
~5.6 mJy at 34 GHz.

Analysis is ongoing and subsequent epochs are planned.

Thank you to CSIRO staff for supporting these observations.
\end{verbatim}

\ateltitle{11818}{AT2018cow: Further ATCA monitoring}
\begin{center}
\textit{ Dougal Dobie (University of Sydney), Vikram Ravi (Caltech), Anna Ho (Caltech),\\
Mansi Kasliwal (Caltech), Tara Murphy (University of Sydney)\\
on 5 Jul 2018; 05:33 UT\\
Credential Certification: Dougal Dobie (ddob1600@uni.sydney.edu.au)}
\end{center}
\begin{verbatim}
We have observed the position of AT2018cow (ATel #11727) with the
Australia Telescope Compact Array (ATCA) on 2018-06-28 from 08:30-14:00 UT
at central frequencies of 5.5 GHz, 9 GHz, 16.7 GHz and 21.2 GHz with a
bandwidth of 2 GHz in each band, and 34 GHz with a bandwidth of 4 GHz.

A point source coincident with the location of AT2018cow (ATel #11727)
is detected in all bands. We report preliminary flux densities of:

~0.22 mJy at 5.5 GHz;
~0.52 mJy at 9 GHz;
~1.5 mJy at 16.7 GHz;
~2.3 mJy at 21.2 GHz;
~7.6 mJy at 34 GHz.

Analysis is ongoing and subsequent epochs are planned.

A flux-density scale error in our previous ATCA observations (ATel #11795)
resulted in an incorrect measurement being reported at 9 GHz. The corrected
flux density of AT2018cow at 9 GHz on 2018-06-26 is ~0.3 mJy.

Thank you to CSIRO staff for supporting these observations.
\end{verbatim}

\pagebreak
\ateltitle{11862}{AT2018cow: Continued ATCA monitoring}
\begin{center}
\textit{ Dougal Dobie (University of Sydney), Vikram Ravi (Caltech), Anna Ho (Caltech),\\
Mansi Kasliwal (Caltech), Tara Murphy (University of Sydney)\\
on 17 Jul 2018; 04:29 UT\\
Credential Certification: Dougal Dobie (ddob1600@uni.sydney.edu.au)}
\end{center}
\begin{verbatim}
We have observed the position of AT2018cow (ATel #11727) with the Australia
Telescope Compact Array (ATCA) on 2018-07-03 between 09:00-13:45 UT at central
frequencies of 5.5 GHz and 9 GHz with a bandwidth of 2 GHz in each band.

We report preliminary flux densities of ~0.4 mJy at 5.5 GHz and ~1.0 mJy at 9 GHz.

We have also observed AT2018cow on 2018-07-05 from 13:30-15:30 UT at a central
frequency of 34 GHz with a bandwidth of 4 GHz.

We report a preliminary flux density of ~10 mJy.

Analysis is ongoing and subsequent epochs are planned.

Thank you to CSIRO staff for supporting these observations.
\end{verbatim}

\ateltitle{12981}{ASKAP observations of blazars possibly associated with neutrino events IC190730A and IC190704A}
\begin{center}
\textit{Dougal Dobie (University of Sydney/CSIRO), David L. Kaplan (University of Wisconsin, Milwaukee), Adam Stewart (University of Sydney), Tara Murphy (University of Sydney),\\
Emil Lenc (CSIRO), David McConnell (CSIRO), Aidan Hotan (CSIRO),\\
Julie Banfield (CSIRO), Wasim Raja (CSIRO), Matthew Whiting (CSIRO)\\
on 2 Aug 2019; 07:23 UT\\
Credential Certification: Dougal Dobie (ddob1600@uni.sydney.edu.au)}
\end{center}
\begin{verbatim}
GCN #24981 reports the detection of a high-energy astrophysical neutrino candidate,
IC190704A, spatially coincident with the known blazar 1WHSP J104516.2+275133.

ATel #12967 reports the detection of a high-energy astrophysical neutrino candidate,
IC190730A, spatially coincident with the known blazar PKS 1502+106.

We have obtained observations of both blazars prior to the detection of the
corresponding neutrino candidates with the Australian Square Kilometre Array
Pathfinder (ASKAP) as part of the Rapid ASKAP Continuum Survey (RACS).
Observations were carried out at 888 MHz with a bandwidth of 288 MHz.

The field containing PKS 1502+106 was observed on 2019-04-24 and we find a flux
density of ~1.68 Jy, which is consistent with archival measurements. Historically
this source appears to exhibit minimal radio variability.

Two fields containing 1WHSP J104516.2+275133 were observed on 2019-04-21 and we find
a flux density of ~4.5 mJy in each. This is broadly consistent with observations
with the VLA on 2019-07-09 (ATel #12926) and archival data from FIRST. Comparing
the RACS measurement to the observation as part of the VLA Sky Survey on 2019-06-08
suggests the blazar had a much steeper spectrum prior to the neutrino event and
is exhibiting significant variability.

This work was done as part of the ASKAP Variables and Slow Transients (VAST)
collaboration (Murphy et al. 2013, PASA, 30, 6).
\end{verbatim}
\pagebreak
\ateltitle{13403}{ATCA observations of SN2020oi}
\begin{center}
\textit{ Dougal Dobie (University of Sydney/CSIRO), Assaf Horesh (HUJI),\\
Andrew O'Brien (University of Wisconsin, Milwaukee), Tara Murphy (University of Sydney)\\
on 14 Jan 2020; 08:44 UT\\
Credential Certification: Dougal Dobie (ddob1600@uni.sydney.edu.au)}
\end{center}
\begin{verbatim}
We have performed preliminary analysis of observations of SN2020oi (TNS Report
58241) with the Australia Telescope Compact Array on 2020-01-11.

We report the detection of a point source spatially coincident with the supernova
with a flux density of ~1 mJy at 21.2 GHz.

Analysis is ongoing and further observations are planned.

Thank you to CSIRO staff for supporting these observations.
\end{verbatim}

\endgroup