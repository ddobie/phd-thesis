\begin{table}[b!]
\centering
\resizebox{\linewidth}{!}{
    \begin{threeparttable}
    \begin{tabular}{clcccccc}
    \hline\hline
    Epoch & SBID & Start & Int. time & $\Delta$T & \% Flagged & Sensitivity & Beam Size\\
     & & (UTC) & (h:m:s) & (d) & & ($\mu$Jy)\\
    \hline
    0 & 8582 & 2019-04-27 04:59:14 & 00:15:00 & $-$110 & 26  & 270 & $10.2\arcsec\times14.9\arcsec$\\
    1 & 9602 & 2019-08-16 14:10:27 & 10:39:25 & 2 & 25 & 35  & $10.0\arcsec\times12.3\arcsec$\\
    2 & 9649 & 2019-08-23 13:42:59 & 10:39:01 & 9 & 26 & 39  & $11.8\arcsec\times12.4\arcsec$\\
    3 & 9910 & 2019-09-16 12:08:34 & 10:38:42 & 33 & 32 & 39  & $9.8\arcsec\times12.1\arcsec$\\
    \hline\hline
    \end{tabular}
    \end{threeparttable}}
    \captionsetup{width=\linewidth}
    \caption[ASKAP Observations of GW190814]{Details of our ASKAP observations for each scheduling block ID (SBID). All observations were carried out with 288\,MHz of bandwidth centered on a frequency of 944\,MHz and 33 of 36 antennas. Typically 26\% of the data was flagged due to RFI or correlator drop-outs. The ASKAP images from our follow-up observations are available from the CSIRO ASKAP Science Data Archive\footnote{\url{https://casda.csiro.au/}} under project code AS111.}
\label{tab:obs_descrip}
\end{table}
