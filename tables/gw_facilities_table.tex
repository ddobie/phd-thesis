\begin{table}
    \caption[Current and future gravitational wave detector network specifications]{Capabilities of gravitational wave detector networks made of the Hanford (H), Livingston (L), Virgo (V), Kagra (K), LIGO-India (I) detectors. Detectors improved by the A+ upgrade are denoted by a subscript $+$ while LIGO-Voyager detectors are denoted by a subscript $V$.}
    \label{tab:gw_facilities}
    \centering
    \begin{threeparttable}
    \begin{tabular}{llccc}
    \hline\hline
    Epoch & Facilities & Timeline & Range\tnote{a} & Localisation\tnote{b}\\
     & & & (Mpc) & ($\deg^2$)\\
    \hline
    O4 & H, L, V, K & 2022--2023 & 190 & 35 \\
    O5 & H$_{+}$, L$_{+}$, V$_{+}$, K & 2025--2026 & 330 & 35 \\
    2G & H$_{+}$, L$_{+}$, V$_{+}$, K, I$_{+}$ & 2026 & 330 & 35\\
    \rule{0pt}{3ex}Voyager & H$_{\rm V}$, L$_{\rm V}$, V$_{\rm V}$ & 2030 & 1100 & 70\\
    \rule{0pt}{3ex}3G & ET, CE, Voy & 2040 & $10^{5}$ & 10\\
    ~ & ET, 2CE & & $10^{5}$ & 1\\
    \hline\hline
    \end{tabular}
    \begin{tablenotes}\footnotesize
    \item[a] Maximum range of any detector in the network
    \item[b] Order of magnitude estimate for typical localisation
    \end{tablenotes}
    \end{threeparttable}
\end{table}