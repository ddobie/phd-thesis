\phantomsection\addcontentsline{toc}{chapter}{Abstract} % Don't remove me!
\chapter*{\centering{\Large Abstract}}

I present comprehensive analysis of the role of radio observations in the field of multi-messenger astronomy. I demonstrate evidence for a turnover in the radio lightcurve of GW170817, the first detection of a neutron star merger. I propose an optimised observation strategy for follow-up of gravitational wave events with widefield radio telescopes, and apply this strategy to follow-up of a possible neutron star-black hole merger, GW190814. I discuss prospects of using Very Long Baseline Interferometry, and high-cadence observations of scintillation-induced variability, to constrain properties of the relativistic outflows produced by mergers. I quantify the capability of existing and planned radio facilities to perform gravitational wave follow-up and monitoring of detected afterglows. I conclude by summarising this work and place it in the broader context of the multi-messenger era.