\chapter{Conclusion}
\label{chap:conclusion}
\vspace{20pt}
\begingroup

The first detection of gravitational waves and light from a neutron star merger, GW170817, heralds the dawn of the era of multi-messenger astronomy \citep{2017ApJ...848L..12A}. This discovery confirmed the association between short GRBs and neutron star mergers \citep{2017ApJ...848L..13A}, the origin of the heavy elements \citep{2017ApJ...848L..19C,2017Sci...358.1559K,2017Natur.551...67P} and that gravitational waves propagate at the speed of light \citep{2017ApJ...848L..13A}. It also enabled independent measurements of the Hubble constant \citep{2017Natur.551...85A,2019NatAs...3..940H} and various tests of General Relativity \citep{2019PhRvL.123a1102A}.

These insights were only possible due to comprehensive follow-up observations with telescopes across the entire electromagnetic spectrum. The gamma-ray emission can be used to probe the central engine of the merger, and compare the arrival time of gravitational waves and light. The optical emission is dependent on factors including the opacity of the surrounding material and the mass of the ejecta and was vital in localising the merger. Radio observations detect synchrotron emission produced by the relativistic outflow from the merger, provide unique insight into the geometry and dynamics of the outflow and the surrounding environment. As part of this thesis I carried out observations and analysis of the radio afterglow, including the confirmation that the emission had peaked and begun to decline (Chapter \ref{chap:gw170817_turnover}).

While observations of GW170817 have answered many questions, there is a limit to the knowledge that can be gained with a sample size of one. Observations of a larger sample of mergers will enable studies of the neutron star merger population. This will lead to more breakthroughs, including a better understanding of the relationship between short GRBs and neutron star mergers, tighter constraints on the expected rate of electromagnetic transients produced by neutron star mergers, and more precise standard-siren measurements of the Hubble constant.

Progress towards a larger sample of mergers will be spurred by improvements in both gravitational wave detectors and follow-up capabilities.

\clearpage
\pagebreak

Planned upgrades to existing gravitational wave detectors, along with the commissioning of new ones, will improve the sensitivity of the network \citep{2018LRR....21....3A,2019CQGra..36v5002H}. Not only will this expand the merger detection horizon and increase the rate of detections, but it will also improve the localisation of events. Smaller localisation regions will facilitate more efficient follow-up observations, and enable the discovery of electromagnetic counterparts to become commonplace.

The commissioning of new electromagnetic facilities will also improve our localisation capabilities and allow for deeper study of the properties of afterglows. The Transient High-Energy Sky and Early Universe Surveyor \citep[THESEUS;][]{2018AdSpR..62..191A} will improve the characterisation of GRBs and enable mergers to be localised to arcminute precision in real-time. Optical and infra-red facilities such as the Vera C. Rubin Observatory \citep{2018arXiv181204051M}, the Southern Gravitational-Wave Optical Transient Observer \citep[GOTO;][]{2020MNRAS.497..726G} and the Dynamic REd All-sky Monitoring Survey \citep[DREAMS;][]{2020SPIE11203E..07S} will improve the efficiency of searches for kilonovae. Untargeted surveys for transients with these facilities will also likely detect emission from neutron star mergers independent of any gravitational wave signals \citep{2019PASP..131f8004A}.

However, searches for kilonovae will not be sufficient to localise all mergers. Follow-up of some events may be hindered by observational constraints, or extrinsic factors such as dust obscuration. Mergers of massive neutron star binaries, or neutron star-black holes, may not produce detectable optical or infra-red emission, although radio emission is still expected. Radio telescopes are also subject to far fewer observational constraints than optical facilities, and radio emission is generally not hindered by material along the line of sight. Therefore, radio follow-up is a promising method for localising mergers where facilities operating at other wavelengths cannot. %However, radio follow-up efforts during the first two LIGO observing runs was hindered by a lack of telescopes capable of surveying large areas of sky.

The commissioning of ASKAP and MeerKAT has enabled unprecedented widefield searches for radio counterparts to mergers, and these capabilities will only be improved as facilities like the DSA-2000 and ultimately, the Square Kilometre Array, come online. I have used ASKAP to carry out the first widefield radio follow-up of a gravitational wave event, and placed tight constraints on the inclination angle of the merger and the density of its surrounding environment (Chapter \ref{chap:GW190814}). I have also developed an optimised follow-up strategy that is tailored to widefield radio telescopes (Chapter \ref{chap:optimised_followup}), and quantified prospects for localising mergers with all suitable existing and planned radio facilities (Chapter \ref{chap:radio_gw_detectability}).

ASKAP and MeerKAT will also carry out widefield, GHz-frequency, transients surveys that will begin to probe the expected rate of radio transients for the first time. These facilities will detect the radio afterglows of neutron star mergers without independent of the detection of GRBs or gravitational waves, which are subject to inclination angle selection effects. This will allow measurement of the GRB beaming fraction, and in turn enable estimates of the rate of short GRBs in the Universe, and the true energy scale of the emission they produce.

\clearpage
\pagebreak

The contribution of radio observations to the multi-messenger era is not limited to localisation. Broadband monitoring of the radio afterglow produced by mergers is necessary to constrain properties including the inclination angle, circum-merger density, and energetics of the jet. A large sample of merger afterglows will shed light on the typical microphysics parameters associated with these outflows, and potentially allow study of the properties of the central engine that drives them. However, measurements beyond simple lightcurve monitoring are required to break model degeneracies such as the relationship between the inclination and jet opening angle angles.

This could be achieved via VLBI imaging to directly image outflow structure and detect relativisitc motion, or by using high-cadence monitoring to detect scintillation and infer source sizes. In this thesis I have outlined prospects for these techniques and quantified the population of mergers they will provide useful constraints for (Chapter \ref{chap:neutron_star_merger_outflows}). The inclination angle constraints provided by VLBI observations in particular will be useful in improving the precision of standard siren measurements of the Hubble Constant.

We are entering a new era of transient astronomy. The combination of gravitational wave detectors, the most sensitive widefield optical telescopes ever constructed, and the first widefield radio telescopes will provide an unprecedented view of the dynamic sky. Much like observations of the planets and the stars shaped our notions of the Universe many millennia ago, the transient events that will be detected in the coming years will lead to a deeper understanding of the Universe, and our place in it.

\endgroup